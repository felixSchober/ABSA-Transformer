\chapter{Introduction}
\label{chap:introduction}

\section{Motivation}
\label{sec:introduction_motivation}

die zahl der personen die online reviews schreiben steigt stetig

manche produkte auf e-commerce seiten wie amazon können hunderte oder tausende reviews enthalten

nur ein paar reviews lesen -> time consuming und nur biased 

eingedampft auf sterne wertung
machnche aspekte wichtig und manche aspekte unwichtig. für andere personen anders

good motivation: "Reviews for products online are seldom fully negative or positive in sentiment. Rather, they describe the positive and negative core aspects of a product. To demonstrate this issue, consider an excerpt from this 3/5 star review for a laptop:
“The faux leather cover is a wee bit cheesy for my taste, but I loved the price and the performance.”
For a purchaser, this review may not be useful when viewed only as a contribution to a mean score. The aspects “performance” and “price” have highly positive sentiment, while “appearence” receives a slightly negative sentiment. For a customer indifferent to the aesthetic of a laptop, this review should contribute higher than a 3/5 score to the mean.
More useful to the consumer are summary statistics for each of a product’s features. Our goal is to bring this structure to Amazon product reviews using deep learning." \cite{Marx2015}

"
However, the reader’s taste may differ
from the reviewers’. For example, the reader may feel
strongly about the quality of the gym in a hotel, whereas
many reviewers may focus on other aspects of the ho-tel, such as the decor or the location. Thus, the reader is
forced to wade through a large number of reviews looking
for information about particular features of interest." \cite{Popescu2005}

Example:
"
In order to illustrate the task at hand, let us consider a text snippet expressing a customer’s opinion
about a particular beer. “This beer is tasty and leaves a thick lacing around the glass” This snippet
discusses multiple aspects such as the taste of the beer and its appearance. The review expresses
positive sentiments about both the aspects. It is interesting to note that the word “tasty” serves both
as an aspect as well as a sentiment word in this case. The phrase “leaves a thick lacing” suggests
that the snippet is discussing about the appearance of the beer and usage of “thick lacing” can be
attributed to positive sentiment. This example demonstrates the intricacies involved in the task of
aspect specific sentiment analysis." \cite{Lakkaraju2014}


"interleaving these two phases in a more tightly coupled manner allows us to capture subtle dependencies." \cite{Lakkaraju2014}


"We conclude by examining
factors that make the sentiment classifica-tion problem more challenging." \cite{Pang2012}

it is very expensive to collect annotated data for absa. Therefore models have to be trained with less data.

Sales forecast based on sentiment prediction of customer reviews \cite{Shen2015}

\section{Outline}
